\chapter{Annexe A : Code source principal}

Cette annexe présente les extraits de code source les plus significatifs de notre application.

\section{Configuration de la base de données}

\subsection{Schéma MongoDB - Modèle Utilisateur}

\begin{lstlisting}[language=JavaScript, caption=userModel.js - Modèle utilisateur complet]
import mongoose from 'mongoose';

const userSchema = new mongoose.Schema({
  name: { 
    type: String, 
    required: true,
    trim: true,
    minlength: 2,
    maxlength: 100
  },
  email: { 
    type: String, 
    required: true, 
    unique: true,
    lowercase: true,
    match: [/^\w+([.-]?\w+)*@\w+([.-]?\w+)*(\.\w{2,3})+$/, 'Email invalide']
  },
  password: { 
    type: String, 
    required: true,
    minlength: 6
  },
  image: { 
    type: String, 
    default: 'https://res.cloudinary.com/daxvcwdwb/image/upload/v1234567890/default_avatar.png' 
  },
  address: {
    line1: { type: String, default: '', maxlength: 200 },
    line2: { type: String, default: '', maxlength: 200 }
  },
  gender: { 
    type: String, 
    default: 'Not Selected',
    enum: ['Male', 'Female', 'Not Selected']
  },
  dob: { 
    type: String, 
    default: '2000-01-01',
    match: [/^\d{4}-\d{2}-\d{2}$/, 'Format de date invalide (YYYY-MM-DD)']
  },
  phone: { 
    type: String, 
    default: '000000000',
    match: [/^[0-9]{9,15}$/, 'Numéro de téléphone invalide']
  },
  role: {
    type: String,
    enum: ['citizen', 'agent', 'supervisor', 'admin'],
    default: 'citizen'
  },
  isVerified: {
    type: Boolean,
    default: false
  },
  verificationToken: {
    type: String,
    default: null
  },
  resetPasswordToken: {
    type: String,
    default: null
  },
  resetPasswordExpires: {
    type: Date,
    default: null
  },
  lastLogin: {
    type: Date,
    default: null
  },
  isActive: {
    type: Boolean,
    default: true
  }
}, { 
  minimize: false,
  timestamps: true
});

// Index pour améliorer les performances
userSchema.index({ email: 1 });
userSchema.index({ role: 1 });
userSchema.index({ isActive: 1 });

// Méthodes du modèle
userSchema.methods.toJSON = function() {
  const user = this.toObject();
  delete user.password;
  delete user.verificationToken;
  delete user.resetPasswordToken;
  delete user.resetPasswordExpires;
  return user;
};

const userModel = mongoose.model('user', userSchema);
export default userModel;
\end{lstlisting}

\section{API Backend - Contrôleurs}

\subsection{Contrôleur de gestion des rendez-vous}

\begin{lstlisting}[language=JavaScript, caption=appointmentController.js - Gestion complète des rendez-vous]
import appointmentModel from '../models/appointmentModel.js';
import doctorModel from '../models/doctorModel.js';
import userModel from '../models/userModel.js';
import { sendConfirmationEmail, sendCancellationEmail } from '../services/emailService.js';

// Réserver un rendez-vous avec validation complète
const bookAppointment = async (req, res) => {
  try {
    const { userId, docId, slotDate, slotTime } = req.body;
    
    // Validation des paramètres
    if (!userId || !docId || !slotDate || !slotTime) {
      return res.json({
        success: false,
        message: 'Tous les champs sont obligatoires'
      });
    }
    
    // Vérifier que l'utilisateur existe
    const userData = await userModel.findById(userId).select('-password');
    if (!userData || !userData.isActive) {
      return res.json({
        success: false,
        message: 'Utilisateur non trouvé ou inactif'
      });
    }
    
    // Vérifier que le service/agent existe et est disponible
    const docData = await doctorModel.findById(docId).select('-password');
    if (!docData) {
      return res.json({
        success: false,
        message: 'Service non trouvé'
      });
    }
    
    if (!docData.available) {
      return res.json({
        success: false,
        message: 'Service temporairement indisponible'
      });
    }
    
    // Vérifier que la date n'est pas dans le passé
    const selectedDate = new Date(slotDate.replace(/_/g, '/'));
    const today = new Date();
    today.setHours(0, 0, 0, 0);
    
    if (selectedDate < today) {
      return res.json({
        success: false,
        message: 'Impossible de réserver dans le passé'
      });
    }
    
    // Vérifier la disponibilité du créneau
    let slots_booked = docData.slots_booked || {};
    
    if (slots_booked[slotDate] && slots_booked[slotDate].includes(slotTime)) {
      return res.json({
        success: false,
        message: 'Ce créneau n\'est plus disponible'
      });
    }
    
    // Vérifier que l'utilisateur n'a pas déjà un rendez-vous à ce créneau
    const existingAppointment = await appointmentModel.findOne({
      userId,
      slotDate,
      slotTime,
      cancelled: false
    });
    
    if (existingAppointment) {
      return res.json({
        success: false,
        message: 'Vous avez déjà un rendez-vous à ce créneau'
      });
    }
    
    // Mettre à jour les créneaux réservés
    if (!slots_booked[slotDate]) {
      slots_booked[slotDate] = [];
    }
    slots_booked[slotDate].push(slotTime);
    
    // Créer le rendez-vous
    const appointmentData = {
      userId,
      docId,
      userData: {
        name: userData.name,
        email: userData.email,
        phone: userData.phone,
        address: userData.address
      },
      docData: {
        name: docData.name,
        department: docData.speciality,
        email: docData.email,
        image: docData.image
      },
      amount: docData.fees || 0,
      slotTime,
      slotDate,
      date: Date.now(),
      status: 'scheduled',
      createdBy: 'user'
    };
    
    const newAppointment = new appointmentModel(appointmentData);
    await newAppointment.save();
    
    // Mettre à jour les créneaux réservés du service
    await doctorModel.findByIdAndUpdate(docId, { 
      slots_booked,
      $inc: { appointmentsCount: 1 }
    });
    
    // Envoyer l'email de confirmation
    try {
      await sendConfirmationEmail(userData.email, appointmentData);
    } catch (emailError) {
      console.log('Erreur envoi email:', emailError);
      // Ne pas faire échouer la réservation pour un problème d'email
    }
    
    // Log de l'action pour audit
    console.log(`Rendez-vous créé: User ${userId} - Service ${docId} - Date ${slotDate} ${slotTime}`);
    
    res.json({
      success: true,
      message: 'Rendez-vous réservé avec succès',
      appointmentId: newAppointment._id
    });
    
  } catch (error) {
    console.log('Erreur booking appointment:', error);
    res.json({
      success: false,
      message: 'Erreur serveur. Veuillez réessayer.'
    });
  }
};

// Annuler un rendez-vous
const cancelAppointment = async (req, res) => {
  try {
    const { userId, appointmentId } = req.body;
    
    // Trouver le rendez-vous
    const appointment = await appointmentModel.findById(appointmentId);
    
    if (!appointment) {
      return res.json({
        success: false,
        message: 'Rendez-vous non trouvé'
      });
    }
    
    // Vérifier que l'utilisateur peut annuler ce rendez-vous
    if (appointment.userId !== userId) {
      return res.json({
        success: false,
        message: 'Vous n\'êtes pas autorisé à annuler ce rendez-vous'
      });
    }
    
    if (appointment.cancelled) {
      return res.json({
        success: false,
        message: 'Ce rendez-vous est déjà annulé'
      });
    }
    
    // Vérifier que le rendez-vous peut encore être annulé (ex: pas moins de 2h avant)
    const appointmentDateTime = new Date(appointment.slotDate.replace(/_/g, '/') + ' ' + appointment.slotTime);
    const now = new Date();
    const timeDiff = appointmentDateTime.getTime() - now.getTime();
    const hoursDiff = timeDiff / (1000 * 3600);
    
    if (hoursDiff < 2) {
      return res.json({
        success: false,
        message: 'Impossible d\'annuler moins de 2 heures avant le rendez-vous'
      });
    }
    
    // Marquer comme annulé
    await appointmentModel.findByIdAndUpdate(appointmentId, {
      cancelled: true,
      cancelledAt: new Date(),
      cancelledBy: 'user'
    });
    
    // Libérer le créneau
    const docData = await doctorModel.findById(appointment.docId);
    if (docData && docData.slots_booked && docData.slots_booked[appointment.slotDate]) {
      const slotIndex = docData.slots_booked[appointment.slotDate].indexOf(appointment.slotTime);
      if (slotIndex > -1) {
        docData.slots_booked[appointment.slotDate].splice(slotIndex, 1);
        await docData.save();
      }
    }
    
    // Envoyer email de confirmation d'annulation
    try {
      const userData = await userModel.findById(userId).select('-password');
      await sendCancellationEmail(userData.email, appointment);
    } catch (emailError) {
      console.log('Erreur envoi email annulation:', emailError);
    }
    
    res.json({
      success: true,
      message: 'Rendez-vous annulé avec succès'
    });
    
  } catch (error) {
    console.log('Erreur cancel appointment:', error);
    res.json({
      success: false,
      message: 'Erreur serveur'
    });
  }
};

export { bookAppointment, cancelAppointment };
\end{lstlisting}

\section{Frontend React - Composants principaux}

\subsection{Composant de réservation de rendez-vous}

\begin{lstlisting}[language=JavaScript, caption=AppointmentForm.jsx - Formulaire de réservation avec validation]
import React, { useContext, useEffect, useState } from 'react';
import { useNavigate, useParams } from 'react-router-dom';
import { AppContext } from '../context/AppContext';
import { toast } from 'react-hot-toast';
import axios from 'axios';

const AppointmentForm = () => {
  const { docId } = useParams();
  const navigate = useNavigate();
  const { 
    doctors, 
    currencySymbol, 
    backendUrl, 
    token, 
    getDoctorsData,
    userData 
  } = useContext(AppContext);
  
  // États pour la gestion des créneaux
  const [docInfo, setDocInfo] = useState(null);
  const [docSlots, setDocSlots] = useState([]);
  const [slotIndex, setSlotIndex] = useState(0);
  const [slotTime, setSlotTime] = useState('');
  const [loading, setLoading] = useState(false);
  const [selectedDate, setSelectedDate] = useState('');
  
  // Configuration des jours de la semaine
  const daysOfWeek = ['DIM', 'LUN', 'MAR', 'MER', 'JEU', 'VEN', 'SAM'];
  
  // Récupérer les informations du service/agent
  const fetchDocInfo = async () => {
    try {
      const docInfo = doctors.find(doc => doc._id === docId);
      if (docInfo) {
        setDocInfo(docInfo);
      } else {
        toast.error('Service non trouvé');
        navigate('/services');
      }
    } catch (error) {
      console.error('Erreur récupération service:', error);
      toast.error('Erreur lors du chargement');
    }
  };
  
  // Générer les créneaux disponibles
  const getAvailableSlots = async () => {
    if (!docInfo) return;
    
    setDocSlots([]);
    let today = new Date();
    
    for (let i = 0; i < 7; i++) {
      let currentDate = new Date(today);
      currentDate.setDate(today.getDate() + i);
      
      // Définir les heures d'ouverture (8h-17h)
      let endTime = new Date(currentDate);
      endTime.setHours(17, 0, 0, 0);
      
      // Pour aujourd'hui, commencer à l'heure actuelle + 1h
      if (today.getDate() === currentDate.getDate()) {
        currentDate.setHours(
          currentDate.getHours() > 8 ? currentDate.getHours() + 1 : 8
        );
        currentDate.setMinutes(currentDate.getMinutes() > 30 ? 30 : 0);
      } else {
        currentDate.setHours(8);
        currentDate.setMinutes(0);
      }
      
      let timeSlots = [];
      
      // Générer les créneaux de 30 minutes
      while (currentDate < endTime) {
        let formattedTime = currentDate.toLocaleTimeString(
          'fr-FR', 
          { hour: '2-digit', minute: '2-digit', hour12: false }
        );
        
        let day = currentDate.getDate();
        let month = currentDate.getMonth() + 1;
        let year = currentDate.getFullYear();
        
        const slotDate = day + "_" + month + "_" + year;
        
        // Vérifier si le créneau est disponible
        const isSlotBooked = docInfo.slots_booked && 
                            docInfo.slots_booked[slotDate] && 
                            docInfo.slots_booked[slotDate].includes(formattedTime);
        
        // Éviter les créneaux à l'heure du déjeuner (12h-14h)
        const isLunchTime = currentDate.getHours() >= 12 && currentDate.getHours() < 14;
        
        if (!isSlotBooked && !isLunchTime) {
          timeSlots.push({
            datetime: new Date(currentDate),
            time: formattedTime,
            slotDate: slotDate
          });
        }
        
        currentDate.setMinutes(currentDate.getMinutes() + 30);
      }
      
      setDocSlots(prev => ([...prev, timeSlots]));
    }
  };
  
  // Valider la sélection avant réservation
  const validateSelection = () => {
    if (!token) {
      toast.error('Veuillez vous connecter pour réserver');
      navigate('/login');
      return false;
    }
    
    if (!userData) {
      toast.error('Profil utilisateur non chargé');
      return false;
    }
    
    if (!slotTime) {
      toast.error('Veuillez sélectionner un créneau horaire');
      return false;
    }
    
    if (!selectedDate) {
      toast.error('Veuillez sélectionner une date');
      return false;
    }
    
    return true;
  };
  
  // Réserver le rendez-vous
  const bookAppointment = async () => {
    if (!validateSelection()) return;
    
    setLoading(true);
    
    try {
      const selectedSlot = docSlots[slotIndex]?.find(slot => slot.time === slotTime);
      
      if (!selectedSlot) {
        toast.error('Créneau non valide');
        return;
      }
      
      const { data } = await axios.post(
        backendUrl + '/api/user/book-appointment', 
        { 
          docId, 
          slotDate: selectedSlot.slotDate, 
          slotTime 
        }, 
        { 
          headers: { token },
          timeout: 10000
        }
      );
      
      if (data.success) {
        toast.success(data.message);
        getDoctorsData(); // Rafraîchir les données
        
        // Rediriger vers la page des rendez-vous après un délai
        setTimeout(() => {
          navigate('/my-appointments');
        }, 2000);
      } else {
        toast.error(data.message);
      }
      
    } catch (error) {
      console.error('Erreur réservation:', error);
      
      if (error.code === 'ECONNABORTED') {
        toast.error('Délai d\'attente dépassé. Veuillez réessayer.');
      } else if (error.response?.status === 429) {
        toast.error('Trop de tentatives. Attendez quelques minutes.');
      } else {
        toast.error('Erreur lors de la réservation. Veuillez réessayer.');
      }
    } finally {
      setLoading(false);
    }
  };
  
  // Sélectionner une date
  const handleDateSelection = (index) => {
    setSlotIndex(index);
    setSlotTime(''); // Réinitialiser le créneau sélectionné
    
    if (docSlots[index] && docSlots[index].length > 0) {
      const date = docSlots[index][0].datetime;
      setSelectedDate(date.toLocaleDateString('fr-FR'));
    }
  };
  
  // Effects
  useEffect(() => {
    fetchDocInfo();
  }, [doctors, docId]);
  
  useEffect(() => {
    if (docInfo) {
      getAvailableSlots();
    }
  }, [docInfo]);
  
  // Rendu conditionnel si pas de données
  if (!docInfo) {
    return (
      <div className="flex justify-center items-center min-h-screen">
        <div className="animate-spin rounded-full h-12 w-12 border-b-2 border-primary"></div>
      </div>
    );
  }
  
  return (
    <div className="max-w-6xl mx-auto p-4">
      {/* Informations du service */}
      <div className="flex flex-col sm:flex-row gap-4 mb-8">
        <div className="flex-shrink-0">
          <img 
            className="bg-primary w-full sm:max-w-72 rounded-lg object-cover h-72 sm:h-auto" 
            src={docInfo.image} 
            alt={docInfo.name}
            onError={(e) => {
              e.target.src = '/default-service.png';
            }}
          />
        </div>
        
        <div className="flex-1 border border-gray-300 rounded-lg p-6 bg-white">
          <div className="flex items-center gap-2 mb-3">
            <h1 className="text-2xl font-bold text-gray-900">{docInfo.name}</h1>
            <div className={`px-2 py-1 rounded-full text-xs ${docInfo.available ? 'bg-green-100 text-green-800' : 'bg-red-100 text-red-800'}`}>
              {docInfo.available ? 'Disponible' : 'Indisponible'}
            </div>
          </div>
          
          <div className="flex items-center gap-2 text-sm text-gray-600 mb-4">
            <span className="font-medium">{docInfo.degree || 'Service'}</span>
            <span>•</span>
            <span>{docInfo.speciality}</span>
            {docInfo.experience && (
              <>
                <span>•</span>
                <span className="bg-blue-100 text-blue-800 px-2 py-0.5 rounded">
                  {docInfo.experience}
                </span>
              </>
            )}
          </div>
          
          <div className="mb-4">
            <h3 className="font-semibold text-gray-900 mb-2">À propos</h3>
            <p className="text-sm text-gray-700 leading-relaxed">
              {docInfo.about || 'Service municipal de qualité pour tous les citoyens d\'Azrou.'}
            </p>
          </div>
          
          <div className="flex items-center gap-4 text-sm">
            {docInfo.fees && (
              <div>
                <span className="text-gray-600">Frais: </span>
                <span className="font-semibold text-green-600">
                  {currencySymbol}{docInfo.fees}
                </span>
              </div>
            )}
            {docInfo.address && (
              <div>
                <span className="text-gray-600">Localisation: </span>
                <span className="font-medium">{docInfo.address}</span>
              </div>
            )}
          </div>
        </div>
      </div>
      
      {/* Sélection des créneaux */}
      <div className="bg-white rounded-lg border border-gray-300 p-6">
        <h2 className="text-lg font-semibold text-gray-900 mb-4">
          Sélectionner un créneau
        </h2>
        
        {docSlots.length > 0 ? (
          <>
            {/* Sélection de la date */}
            <div className="mb-6">
              <h3 className="text-sm font-medium text-gray-700 mb-3">
                Choisir une date
              </h3>
              <div className="flex gap-2 overflow-x-auto pb-2">
                {docSlots.map((daySlots, index) => {
                  const day = daySlots[0]?.datetime;
                  if (!day) return null;
                  
                  return (
                    <button
                      key={index}
                      onClick={() => handleDateSelection(index)}
                      className={`flex flex-col items-center min-w-[60px] p-3 rounded-lg border transition-all ${
                        slotIndex === index 
                          ? 'bg-primary text-white border-primary' 
                          : 'bg-gray-50 text-gray-700 border-gray-200 hover:border-primary'
                      }`}
                    >
                      <span className="text-xs font-medium">
                        {daysOfWeek[day.getDay()]}
                      </span>
                      <span className="text-lg font-bold">
                        {day.getDate()}
                      </span>
                      <span className="text-xs">
                        {day.toLocaleDateString('fr-FR', { month: 'short' })}
                      </span>
                    </button>
                  );
                })}
              </div>
            </div>
            
            {/* Sélection de l'heure */}
            {docSlots[slotIndex] && docSlots[slotIndex].length > 0 && (
              <div className="mb-6">
                <h3 className="text-sm font-medium text-gray-700 mb-3">
                  Choisir un horaire {selectedDate && `pour le ${selectedDate}`}
                </h3>
                <div className="grid grid-cols-3 sm:grid-cols-4 md:grid-cols-6 gap-2">
                  {docSlots[slotIndex].map((slot, index) => (
                    <button
                      key={index}
                      onClick={() => setSlotTime(slot.time)}
                      className={`p-2 rounded-lg text-sm font-medium border transition-all ${
                        slot.time === slotTime
                          ? 'bg-primary text-white border-primary'
                          : 'bg-gray-50 text-gray-700 border-gray-200 hover:border-primary hover:bg-primary hover:text-white'
                      }`}
                    >
                      {slot.time}
                    </button>
                  ))}
                </div>
              </div>
            )}
            
            {/* Bouton de réservation */}
            <div className="flex justify-center">
              <button
                onClick={bookAppointment}
                disabled={!slotTime || loading}
                className={`px-8 py-3 rounded-lg font-medium transition-all ${
                  !slotTime || loading
                    ? 'bg-gray-300 text-gray-500 cursor-not-allowed'
                    : 'bg-primary text-white hover:bg-primary/90'
                }`}
              >
                {loading ? (
                  <div className="flex items-center gap-2">
                    <div className="animate-spin rounded-full h-4 w-4 border-b-2 border-white"></div>
                    Réservation en cours...
                  </div>
                ) : (
                  'Réserver ce créneau'
                )}
              </button>
            </div>
          </>
        ) : (
          <div className="text-center py-8">
            <div className="text-gray-500 mb-2">
              Aucun créneau disponible pour le moment
            </div>
            <button
              onClick={getAvailableSlots}
              className="text-primary hover:underline text-sm"
            >
              Actualiser les créneaux
            </button>
          </div>
        )}
      </div>
    </div>
  );
};

export default AppointmentForm;
\end{lstlisting}

\section{Service de notifications email}

\begin{lstlisting}[language=JavaScript, caption=emailService.js - Service complet de notifications]
import nodemailer from 'nodemailer';
import dotenv from 'dotenv';

dotenv.config();

// Configuration du transporteur email
const createTransporter = () => {
  return nodemailer.createTransporter({
    service: 'gmail',
    auth: {
      user: process.env.EMAIL_USER,
      pass: process.env.EMAIL_PASSWORD
    },
    tls: {
      rejectUnauthorized: false
    }
  });
};

// Template HTML pour les emails
const getEmailTemplate = (type, data) => {
  const baseStyle = `
    <style>
      body { font-family: Arial, sans-serif; line-height: 1.6; color: #333; }
      .container { max-width: 600px; margin: 0 auto; padding: 20px; }
      .header { background: #3B82F6; color: white; padding: 20px; text-align: center; }
      .content { padding: 20px; background: #f9f9f9; }
      .appointment-details { background: white; padding: 15px; border-radius: 5px; margin: 20px 0; }
      .footer { background: #6B7280; color: white; padding: 15px; text-align: center; font-size: 12px; }
      .btn { background: #3B82F6; color: white; padding: 10px 20px; text-decoration: none; border-radius: 5px; display: inline-block; }
    </style>
  `;

  switch (type) {
    case 'confirmation':
      return `
        <!DOCTYPE html>
        <html>
        <head>
          <meta charset="utf-8">
          <title>Confirmation de rendez-vous</title>
          ${baseStyle}
        </head>
        <body>
          <div class="container">
            <div class="header">
              <h1>Commune d'Azrou</h1>
              <h2>Confirmation de rendez-vous</h2>
            </div>
            <div class="content">
              <p>Bonjour <strong>${data.userData.name}</strong>,</p>
              <p>Votre rendez-vous a été confirmé avec succès.</p>
              
              <div class="appointment-details">
                <h3>Détails du rendez-vous :</h3>
                <ul>
                  <li><strong>Service :</strong> ${data.docData.name}</li>
                  <li><strong>Département :</strong> ${data.docData.department}</li>
                  <li><strong>Date :</strong> ${data.slotDate.replace(/_/g, '/')}</li>
                  <li><strong>Heure :</strong> ${data.slotTime}</li>
                  ${data.amount > 0 ? `<li><strong>Frais :</strong> ${data.amount} MAD</li>` : ''}
                </ul>
              </div>
              
              <p><strong>Que faire maintenant ?</strong></p>
              <ul>
                <li>Notez cette date dans votre agenda</li>
                <li>Préparez les documents nécessaires</li>
                <li>Arrivez 10 minutes avant l'heure prévue</li>
                <li>Présentez-vous à l'accueil avec une pièce d'identité</li>
              </ul>
              
              <p><strong>Besoin d'annuler ou reporter ?</strong></p>
              <p>Vous pouvez gérer votre rendez-vous directement sur notre site web ou nous contacter.</p>
              
              <div style="text-align: center; margin: 20px 0;">
                <a href="${process.env.FRONTEND_URL}/my-appointments" class="btn">
                  Gérer mes rendez-vous
                </a>
              </div>
            </div>
            <div class="footer">
              <p>Commune d'Azrou - Service numérique aux citoyens</p>
              <p>Pour toute question : contact@azrou.ma | +212 535-123-000</p>
            </div>
          </div>
        </body>
        </html>
      `;
      
    case 'cancellation':
      return `
        <!DOCTYPE html>
        <html>
        <head>
          <meta charset="utf-8">
          <title>Annulation de rendez-vous</title>
          ${baseStyle}
        </head>
        <body>
          <div class="container">
            <div class="header">
              <h1>Commune d'Azrou</h1>
              <h2>Annulation de rendez-vous</h2>
            </div>
            <div class="content">
              <p>Bonjour <strong>${data.userData.name}</strong>,</p>
              <p>Votre rendez-vous a été annulé avec succès.</p>
              
              <div class="appointment-details">
                <h3>Rendez-vous annulé :</h3>
                <ul>
                  <li><strong>Service :</strong> ${data.docData.name}</li>
                  <li><strong>Date :</strong> ${data.slotDate.replace(/_/g, '/')}</li>
                  <li><strong>Heure :</strong> ${data.slotTime}</li>
                </ul>
              </div>
              
              <p>Si vous souhaitez reprendre rendez-vous, vous pouvez le faire à tout moment sur notre site web.</p>
              
              <div style="text-align: center; margin: 20px 0;">
                <a href="${process.env.FRONTEND_URL}/services" class="btn">
                  Prendre un nouveau rendez-vous
                </a>
              </div>
            </div>
            <div class="footer">
              <p>Commune d'Azrou - Service numérique aux citoyens</p>
              <p>Pour toute question : contact@azrou.ma | +212 535-123-000</p>
            </div>
          </div>
        </body>
        </html>
      `;
      
    case 'reminder':
      return `
        <!DOCTYPE html>
        <html>
        <head>
          <meta charset="utf-8">
          <title>Rappel de rendez-vous</title>
          ${baseStyle}
        </head>
        <body>
          <div class="container">
            <div class="header">
              <h1>Commune d'Azrou</h1>
              <h2>Rappel de rendez-vous</h2>
            </div>
            <div class="content">
              <p>Bonjour <strong>${data.userData.name}</strong>,</p>
              <p>Ceci est un rappel pour votre rendez-vous prévu <strong>demain</strong>.</p>
              
              <div class="appointment-details">
                <h3>Détails du rendez-vous :</h3>
                <ul>
                  <li><strong>Service :</strong> ${data.docData.name}</li>
                  <li><strong>Date :</strong> ${data.slotDate.replace(/_/g, '/')}</li>
                  <li><strong>Heure :</strong> ${data.slotTime}</li>
                </ul>
              </div>
              
              <p><strong>Conseils pour votre visite :</strong></p>
              <ul>
                <li>Arrivez 10 minutes avant l'heure</li>
                <li>Munissez-vous de votre pièce d'identité</li>
                <li>Préparez les documents demandés</li>
                <li>En cas d'empêchement, annulez votre rendez-vous</li>
              </ul>
            </div>
            <div class="footer">
              <p>Commune d'Azrou - Service numérique aux citoyens</p>
              <p>Pour toute question : contact@azrou.ma | +212 535-123-000</p>
            </div>
          </div>
        </body>
        </html>
      `;
      
    default:
      return '';
  }
};

// Envoyer email de confirmation
export const sendConfirmationEmail = async (userEmail, appointmentData) => {
  try {
    const transporter = createTransporter();
    
    const mailOptions = {
      from: `"Commune d'Azrou" <${process.env.EMAIL_USER}>`,
      to: userEmail,
      subject: 'Confirmation de votre rendez-vous - Commune d\'Azrou',
      html: getEmailTemplate('confirmation', appointmentData)
    };
    
    const result = await transporter.sendMail(mailOptions);
    console.log('Email de confirmation envoyé:', result.messageId);
    return result;
    
  } catch (error) {
    console.error('Erreur envoi email confirmation:', error);
    throw error;
  }
};

// Envoyer email d'annulation
export const sendCancellationEmail = async (userEmail, appointmentData) => {
  try {
    const transporter = createTransporter();
    
    const mailOptions = {
      from: `"Commune d'Azrou" <${process.env.EMAIL_USER}>`,
      to: userEmail,
      subject: 'Annulation de votre rendez-vous - Commune d\'Azrou',
      html: getEmailTemplate('cancellation', appointmentData)
    };
    
    const result = await transporter.sendMail(mailOptions);
    console.log('Email d\'annulation envoyé:', result.messageId);
    return result;
    
  } catch (error) {
    console.error('Erreur envoi email annulation:', error);
    throw error;
  }
};

// Envoyer email de rappel
export const sendReminderEmail = async (userEmail, appointmentData) => {
  try {
    const transporter = createTransporter();
    
    const mailOptions = {
      from: `"Commune d'Azrou" <${process.env.EMAIL_USER}>`,
      to: userEmail,
      subject: 'Rappel : Votre rendez-vous de demain - Commune d\'Azrou',
      html: getEmailTemplate('reminder', appointmentData)
    };
    
    const result = await transporter.sendMail(mailOptions);
    console.log('Email de rappel envoyé:', result.messageId);
    return result;
    
  } catch (error) {
    console.error('Erreur envoi email rappel:', error);
    throw error;
  }
};

// Job automatique pour les rappels (à exécuter quotidiennement)
export const sendDailyReminders = async () => {
  try {
    // Calculer la date de demain
    const tomorrow = new Date();
    tomorrow.setDate(tomorrow.getDate() + 1);
    
    const day = tomorrow.getDate();
    const month = tomorrow.getMonth() + 1;
    const year = tomorrow.getFullYear();
    const tomorrowSlotDate = day + "_" + month + "_" + year;
    
    // Trouver tous les rendez-vous de demain
    const tomorrowAppointments = await appointmentModel.find({
      slotDate: tomorrowSlotDate,
      cancelled: false,
      isCompleted: false
    });
    
    console.log(`${tomorrowAppointments.length} rappels à envoyer pour ${tomorrowSlotDate}`);
    
    // Envoyer les rappels
    for (const appointment of tomorrowAppointments) {
      try {
        await sendReminderEmail(appointment.userData.email, appointment);
        await new Promise(resolve => setTimeout(resolve, 1000)); // Délai entre envois
      } catch (error) {
        console.error(`Erreur rappel pour appointment ${appointment._id}:`, error);
      }
    }
    
    console.log('Rappels quotidiens envoyés avec succès');
    
  } catch (error) {
    console.error('Erreur envoi rappels quotidiens:', error);
  }
};
\end{lstlisting}

Ce code source présente les éléments techniques clés de notre application, notamment :

\begin{itemize}
    \item La structure complète des modèles de données MongoDB
    \item Les contrôleurs backend avec gestion d'erreurs robuste
    \item Les composants React avec validation et gestion d'état
    \item Le service de notifications email avec templates HTML
\end{itemize}

Ces extraits illustrent les bonnes pratiques adoptées en termes de sécurité, validation des données, gestion d'erreurs et expérience utilisateur.
