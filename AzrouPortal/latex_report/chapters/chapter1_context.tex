\chapter{Contexte général et présentation de l'organisme d'accueil}

\section{Introduction}

Ce chapitre présente le cadre général dans lequel s'inscrit notre projet, en décrivant d'abord le contexte de la transformation numérique des services publics, puis en présentant la commune d'Azrou comme organisme d'accueil.

\section{Contexte de la transformation numérique des services publics}

\subsection{L'administration électronique au Maroc}

Le Maroc s'est engagé dans une stratégie ambitieuse de modernisation de son administration publique à travers le numérique. Cette transformation s'appuie sur plusieurs piliers :

\begin{itemize}
    \item La stratégie nationale Maroc Digital 2020
    \item Le programme national de simplification des procédures administratives
    \item L'initiative "Administration sans papier"
    \item Le développement de portails de services en ligne
\end{itemize}

\subsection{Enjeux de la digitalisation au niveau communal}

Les collectivités territoriales, et particulièrement les communes, sont au cœur de cette transformation car elles :
\begin{itemize}
    \item Sont les plus proches des citoyens
    \item Gèrent des services essentiels du quotidien
    \item Peuvent avoir un impact direct sur la satisfaction citoyenne
    \item Contribuent à l'amélioration de l'image de l'administration publique
\end{itemize}

\section{Présentation de la commune d'Azrou}

\subsection{Situation géographique et démographique}

\begin{figure}[h]
    \centering
    % \includegraphics[width=0.8\textwidth]{images/azrou_map.png}
    \caption{Localisation géographique d'Azrou}
    \label{fig:azrou_map}
\end{figure}

La commune d'Azrou est située dans la région de Fès-Meknès, dans le Moyen Atlas marocain. Avec une population d'environ 60 000 habitants, elle constitue un centre administratif et économique important de la région.

\textbf{Caractéristiques démographiques :}
\begin{itemize}
    \item Population urbaine : 45 000 habitants
    \item Population rurale : 15 000 habitants
    \item Superficie : 22 km²
    \item Densité : 2 727 habitants/km²
\end{itemize}

\subsection{Organisation administrative}

\subsubsection{Structure organisationnelle}

La commune d'Azrou est organisée en plusieurs départements :

\begin{table}[h]
\centering
\begin{tabular}{|l|l|p{6cm}|}
\hline
\textbf{Département} & \textbf{Personnel} & \textbf{Services principaux} \\
\hline
État Civil & 5 agents & Actes de naissance, mariage, décès \\
\hline
Urbanisme & 8 agents & Permis de construire, certificats d'urbanisme \\
\hline
Finances & 6 agents & Taxes, redevances, aide sociale \\
\hline
Travaux Publics & 12 agents & Voirie, éclairage, assainissement \\
\hline
Services Sociaux & 4 agents & Aide sociale, programmes communautaires \\
\hline
Environnement & 3 agents & Gestion des déchets, espaces verts \\
\hline
\end{tabular}
\caption{Organisation des départements municipaux}
\label{tab:departments}
\end{table}

\subsubsection{Processus actuels de prise de rendez-vous}

Actuellement, la prise de rendez-vous suit le processus suivant :
\begin{enumerate}
    \item Déplacement physique du citoyen à la commune
    \item File d'attente au service d'accueil
    \item Consultation manuelle du planning par l'agent
    \item Attribution d'un créneau disponible
    \item Inscription manuelle sur un registre papier
\end{enumerate}

\subsection{Défis et opportunités}

\subsubsection{Défis identifiés}
\begin{itemize}
    \item Saturation aux heures de pointe
    \item Temps d'attente élevés pour les citoyens
    \item Gestion manuelle sujette aux erreurs
    \item Manque de traçabilité des demandes
    \item Difficultés de planification des ressources
\end{itemize}

\subsubsection{Opportunités de digitalisation}
\begin{itemize}
    \item Amélioration de l'expérience citoyenne
    \item Optimisation des ressources humaines
    \item Réduction des coûts opérationnels
    \item Amélioration de la transparence
    \item Collecte de données pour l'aide à la décision
\end{itemize}

\section{Analyse de l'existant technologique}

\subsection{Infrastructure informatique actuelle}

La commune d'Azrou dispose d'une infrastructure informatique de base :
\begin{itemize}
    \item Réseau local connectant les différents services
    \item Connexion Internet haut débit
    \item Serveur de fichiers centralisé
    \item Postes de travail dans chaque département
\end{itemize}

\subsection{Applications existantes}

Quelques applications sont déjà utilisées :
\begin{itemize}
    \item Logiciel de comptabilité publique
    \item Application de gestion de l'état civil
    \item Système de paie du personnel
\end{itemize}

\subsection{Niveau de maturité numérique}

L'évaluation du niveau de maturité numérique révèle :
\begin{itemize}
    \item \textbf{Points forts} : Infrastructure de base disponible, personnel motivé
    \item \textbf{Points faibles} : Manque d'applications métiers intégrées, processus encore largement manuels
    \item \textbf{Opportunités} : Volonté politique de modernisation, budget alloué au numérique
\end{itemize}

\section{Conclusion}

Cette analyse du contexte montre que la commune d'Azrou présente toutes les caractéristiques nécessaires pour réussir un projet de digitalisation des services de prise de rendez-vous. L'infrastructure de base existe, le personnel est motivé, et la direction politique soutient cette transformation.

Le chapitre suivant présentera l'analyse détaillée des besoins et la spécification des exigences pour notre solution.
