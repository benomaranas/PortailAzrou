\chapter{Analyse des besoins et spécification des exigences}

\section{Introduction}

Ce chapitre présente l'analyse approfondie des besoins des différents acteurs du système et la spécification des exigences fonctionnelles et non-fonctionnelles. Cette analyse constitue la base pour la conception de notre solution.

\section{Méthodologie d'analyse des besoins}

\subsection{Approche adoptée}

Notre approche d'analyse des besoins s'appuie sur :
\begin{itemize}
    \item Des entretiens semi-directifs avec les parties prenantes
    \item L'observation directe des processus existants
    \item L'analyse documentaire des procédures
    \item Des enquêtes auprès des citoyens
\end{itemize}

\subsection{Parties prenantes identifiées}

\begin{table}[h]
\centering
\begin{tabular}{|l|p{8cm}|}
\hline
\textbf{Acteur} & \textbf{Rôle et responsabilités} \\
\hline
Citoyens & Utilisateurs finaux du système, demandeurs de services \\
\hline
Agents d'accueil & Première interface avec les citoyens, gestion des plannings \\
\hline
Personnel des services & Prestataires de services, gestion de leur planning \\
\hline
Superviseurs & Gestion des équipes et validation des demandes \\
\hline
Administrateur système & Gestion technique de l'application \\
\hline
Direction & Suivi des performances et prise de décisions \\
\hline
\end{tabular}
\caption{Identification des parties prenantes}
\label{tab:stakeholders}
\end{table}

\section{Analyse des besoins par acteur}

\subsection{Besoins des citoyens}

\subsubsection{Besoins fonctionnels}
\begin{itemize}
    \item Consulter les services disponibles et leurs modalités
    \item Prendre rendez-vous en ligne à tout moment
    \item Modifier ou annuler un rendez-vous
    \item Recevoir des notifications de confirmation et de rappel
    \item Consulter l'historique de ses rendez-vous
    \item Signaler des problèmes ou faire des suggestions
\end{itemize}

\subsubsection{Besoins non-fonctionnels}
\begin{itemize}
    \item Interface simple et intuitive
    \item Accessibilité sur mobile et desktop
    \item Temps de réponse rapide (< 3 secondes)
    \item Disponibilité 24h/24 et 7j/7
    \item Sécurité des données personnelles
\end{itemize}

\subsection{Besoins du personnel municipal}

\subsubsection{Agents d'accueil}
\textbf{Besoins fonctionnels :}
\begin{itemize}
    \item Consulter les plannings de tous les services
    \item Prendre rendez-vous pour les citoyens (canal téléphonique)
    \item Modifier les rendez-vous en cas d'urgence
    \item Générer des rapports d'activité
\end{itemize}

\textbf{Besoins non-fonctionnels :}
\begin{itemize}
    \item Interface ergonomique adaptée à un usage intensif
    \item Formation minimale requise
    \item Fiabilité du système
\end{itemize}

\subsubsection{Personnel des services}
\textbf{Besoins fonctionnels :}
\begin{itemize}
    \item Consulter leur planning personnel
    \item Modifier leur disponibilité
    \item Accéder aux informations des rendez-vous
    \item Marquer les rendez-vous comme honorés ou manqués
\end{itemize}

\subsection{Besoins des superviseurs et de la direction}

\subsubsection{Besoins fonctionnels}
\begin{itemize}
    \item Consulter les statistiques d'utilisation
    \item Analyser les performances des services
    \item Gérer les comptes utilisateurs et les permissions
    \item Configurer les paramètres du système
    \item Exporter des rapports détaillés
\end{itemize}

\subsubsection{Besoins non-fonctionnels}
\begin{itemize}
    \item Tableaux de bord visuels et informatifs
    \item Capacité d'analyse des tendances
    \item Alertes automatiques en cas d'anomalies
\end{itemize}

\section{Analyse de l'existant et identification des problèmes}

\subsection{Processus actuel de prise de rendez-vous}

\begin{figure}[h]
    \centering
    % Diagramme de processus actuel
    % \includegraphics[width=\textwidth]{images/current_process.png}
    \caption{Processus actuel de prise de rendez-vous}
    \label{fig:current_process}
\end{figure}

\subsection{Problèmes identifiés}

\begin{table}[h]
\centering
\begin{tabular}{|p{4cm}|p{4cm}|p{4cm}|}
\hline
\textbf{Problème} & \textbf{Impact} & \textbf{Fréquence} \\
\hline
Files d'attente longues & Insatisfaction citoyenne & Quotidienne \\
\hline
Doubles réservations & Conflits de planning & Hebdomadaire \\
\hline
Rendez-vous manqués non signalés & Perte de créneaux & Quotidienne \\
\hline
Difficultés de planification & Sous/sur-utilisation des ressources & Mensuelle \\
\hline
Manque de traçabilité & Difficultés de suivi & Permanente \\
\hline
\end{tabular}
\caption{Problèmes identifiés dans le processus actuel}
\label{tab:problems}
\end{table}

\section{Spécification des exigences}

\subsection{Exigences fonctionnelles}

\subsubsection{Module de gestion des rendez-vous}
\begin{itemize}
    \item \textbf{EF-001} : Le système doit permettre aux citoyens de consulter les créneaux disponibles
    \item \textbf{EF-002} : Le système doit permettre la réservation en ligne de rendez-vous
    \item \textbf{EF-003} : Le système doit envoyer une confirmation automatique par email
    \item \textbf{EF-004} : Le système doit permettre la modification et l'annulation de rendez-vous
    \item \textbf{EF-005} : Le système doit envoyer des rappels automatiques
\end{itemize}

\subsubsection{Module d'administration}
\begin{itemize}
    \item \textbf{EF-006} : Le système doit permettre la gestion des utilisateurs et des permissions
    \item \textbf{EF-007} : Le système doit permettre la configuration des services et créneaux
    \item \textbf{EF-008} : Le système doit fournir des tableaux de bord statistiques
    \item \textbf{EF-009} : Le système doit permettre l'export de rapports
    \item \textbf{EF-010} : Le système doit permettre la gestion des problèmes signalés
\end{itemize}

\subsubsection{Module de notifications}
\begin{itemize}
    \item \textbf{EF-011} : Le système doit envoyer des notifications par email
    \item \textbf{EF-012} : Le système doit permettre la configuration des modèles de messages
    \item \textbf{EF-013} : Le système doit enregistrer l'historique des notifications
\end{itemize}

\subsection{Exigences non-fonctionnelles}

\subsubsection{Performance}
\begin{itemize}
    \item \textbf{ENF-001} : Le temps de réponse ne doit pas dépasser 3 secondes
    \item \textbf{ENF-002} : Le système doit supporter 100 utilisateurs concurrent
    \item \textbf{ENF-003} : La disponibilité du système doit être de 99.5\%
\end{itemize}

\subsubsection{Sécurité}
\begin{itemize}
    \item \textbf{ENF-004} : Toutes les données doivent être chiffrées en transit et au repos
    \item \textbf{ENF-005} : L'authentification doit être sécurisée (hachage des mots de passe)
    \item \textbf{ENF-006} : Le système doit respecter le RGPD pour la protection des données
\end{itemize}

\subsubsection{Utilisabilité}
\begin{itemize}
    \item \textbf{ENF-007} : L'interface doit être responsive (mobile, tablette, desktop)
    \item \textbf{ENF-008} : Le système doit être accessible aux personnes handicapées
    \item \textbf{ENF-009} : L'apprentissage du système ne doit pas dépasser 30 minutes
\end{itemize}

\section{Analyse des solutions existantes}

\subsection{Étude comparative des solutions du marché}

\begin{table}[h]
\centering
\scriptsize
\begin{tabular}{|p{2cm}|p{2cm}|p{2cm}|p{2cm}|p{2cm}|}
\hline
\textbf{Solution} & \textbf{Avantages} & \textbf{Inconvénients} & \textbf{Coût} & \textbf{Adaptabilité} \\
\hline
Solution A & Interface moderne & Coût élevé & 50k MAD/an & Faible \\
\hline
Solution B & Open source & Fonctionnalités limitées & Gratuit & Élevée \\
\hline
Solution C & Très complète & Complexité élevée & 80k MAD/an & Moyenne \\
\hline
Solution sur mesure & Parfaitement adaptée & Temps de développement & 120k MAD & Très élevée \\
\hline
\end{tabular}
\caption{Comparaison des solutions existantes}
\label{tab:solutions_comparison}
\end{table}

\subsection{Justification du choix du développement sur mesure}

Le développement d'une solution sur mesure a été retenu pour les raisons suivantes :
\begin{itemize}
    \item Adaptation parfaite aux processus spécifiques de la commune
    \item Possibilité d'évolution et de maintenance
    \item Contrôle total sur la sécurité et les données
    \item Formation et transfert de compétences à l'équipe technique
    \item Coût à long terme plus avantageux
\end{itemize}

\section{Conclusion}

Cette analyse détaillée des besoins nous a permis d'identifier clairement les exigences fonctionnelles et non-fonctionnelles de notre système. Les spécifications établies serviront de base pour la phase de conception qui fait l'objet du chapitre suivant.

Les principales conclusions de cette analyse sont :
\begin{itemize}
    \item Besoin urgent de digitalisation du processus de prise de rendez-vous
    \item Importance de l'expérience utilisateur pour l'adoption du système
    \item Nécessité d'une solution flexible et évolutive
    \item Importance des aspects sécuritaires et de confidentialité
\end{itemize}
