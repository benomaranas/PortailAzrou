\chapter{Conception de la solution}

\section{Introduction}

Ce chapitre présente la conception détaillée de notre solution de gestion des rendez-vous municipaux. Nous y abordons l'architecture globale du système, la modélisation des données, la conception des interfaces utilisateur, et les aspects sécuritaires.

\section{Architecture générale du système}

\subsection{Choix architectural}

Notre solution adopte une architecture en 3 tiers (3-tier architecture) qui sépare clairement :
\begin{itemize}
    \item \textbf{Couche de présentation} : Interface utilisateur (frontend)
    \item \textbf{Couche métier} : Logique applicative (backend)
    \item \textbf{Couche de données} : Base de données
\end{itemize}

\begin{figure}[h]
    \centering
    % \includegraphics[width=0.8\textwidth]{images/system_architecture.png}
    \caption{Architecture générale du système}
    \label{fig:system_architecture}
\end{figure}

\subsection{Technologies retenues}

\begin{table}[h]
\centering
\begin{tabular}{|l|l|p{6cm}|}
\hline
\textbf{Couche} & \textbf{Technologie} & \textbf{Justification} \\
\hline
Frontend & React.js & Composants réutilisables, écosystème riche \\
\hline
Backend & Node.js + Express & Performance, JavaScript full-stack \\
\hline
Base de données & MongoDB & Flexibilité, scalabilité \\
\hline
Authentification & JWT & Stateless, sécurisé \\
\hline
Notifications & Nodemailer & Intégration simple avec Node.js \\
\hline
Déploiement & Vercel / Heroku & Déploiement simplifié \\
\hline
\end{tabular}
\caption{Technologies utilisées par couche}
\label{tab:technologies}
\end{table}

\section{Modélisation des données}

\subsection{Modèle conceptuel de données}

\begin{figure}[h]
    \centering
    % \includegraphics[width=\textwidth]{images/data_model.png}
    \caption{Diagramme de classes du modèle de données}
    \label{fig:data_model}
\end{figure}

\subsection{Description des entités principales}

\subsubsection{Entité User}
\begin{itemize}
    \item \textbf{Attributs} : id, nom, prénom, email, téléphone, adresse, mot de passe, rôle
    \item \textbf{Rôles} : citoyen, agent, superviseur, administrateur
    \item \textbf{Relations} : Un utilisateur peut avoir plusieurs rendez-vous
\end{itemize}

\subsubsection{Entité Department}
\begin{itemize}
    \item \textbf{Attributs} : id, nom, description, services, contact, horaires
    \item \textbf{Relations} : Un département peut avoir plusieurs agents et services
\end{itemize}

\subsubsection{Entité Appointment}
\begin{itemize}
    \item \textbf{Attributs} : id, date, heure, statut, motif, notes
    \item \textbf{Relations} : Associé à un utilisateur et un service
\end{itemize}

\subsubsection{Entité Service}
\begin{itemize}
    \item \textbf{Attributs} : id, nom, description, durée, département
    \item \textbf{Relations} : Appartient à un département
\end{itemize}

\subsection{Schéma de base de données MongoDB}

\begin{lstlisting}[language=JavaScript, caption=Schéma User]
const userSchema = new mongoose.Schema({
  name: { type: String, required: true },
  email: { type: String, required: true, unique: true },
  password: { type: String, required: true },
  phone: String,
  address: String,
  role: { 
    type: String, 
    enum: ['citizen', 'agent', 'supervisor', 'admin'],
    default: 'citizen' 
  },
  isVerified: { type: Boolean, default: false },
  createdAt: { type: Date, default: Date.now }
});
\end{lstlisting}

\section{Conception des interfaces utilisateur}

\subsection{Principes de design}

Notre conception d'interface s'appuie sur les principes suivants :
\begin{itemize}
    \item \textbf{Simplicité} : Interface épurée et intuitive
    \item \textbf{Accessibilité} : Respect des standards WCAG
    \item \textbf{Responsivité} : Adaptation à tous les écrans
    \item \textbf{Cohérence} : Design system uniforme
\end{itemize}

\subsection{Wireframes et maquettes}

\subsubsection{Interface citoyenne}

\begin{figure}[h]
    \centering
    % \includegraphics[width=0.7\textwidth]{images/citizen_wireframe.png}
    \caption{Wireframe de l'interface citoyenne}
    \label{fig:citizen_wireframe}
\end{figure}

Les principales pages pour les citoyens :
\begin{itemize}
    \item Page d'accueil avec présentation des services
    \item Page de connexion/inscription
    \item Page de prise de rendez-vous
    \item Page de gestion des rendez-vous
    \item Page de profil utilisateur
\end{itemize}

\subsubsection{Interface administrative}

\begin{figure}[h]
    \centering
    % \includegraphics[width=0.7\textwidth]{images/admin_wireframe.png}
    \caption{Wireframe de l'interface administrative}
    \label{fig:admin_wireframe}
\end{figure}

Les principales pages pour les administrateurs :
\begin{itemize}
    \item Tableau de bord avec statistiques
    \item Gestion des utilisateurs
    \item Gestion des services et départements
    \item Gestion des rendez-vous
    \item Rapports et analyses
\end{itemize}

\subsection{Charte graphique}

\begin{table}[h]
\centering
\begin{tabular}{|l|l|l|}
\hline
\textbf{Élément} & \textbf{Spécification} & \textbf{Usage} \\
\hline
Couleur primaire & \#3B82F6 (Bleu) & Boutons principaux, liens \\
\hline
Couleur secondaire & \#10B981 (Vert) & Statuts positifs, succès \\
\hline
Couleur d'alerte & \#EF4444 (Rouge) & Erreurs, suppressions \\
\hline
Police principale & Inter & Textes généraux \\
\hline
Police titres & Poppins & Titres et en-têtes \\
\hline
\end{tabular}
\caption{Éléments de la charte graphique}
\label{tab:style_guide}
\end{table}

\section{Architecture logicielle détaillée}

\subsection{Architecture frontend (React.js)}

\subsubsection{Structure des composants}

\begin{lstlisting}[caption=Structure des dossiers frontend]
src/
├── components/          # Composants réutilisables
│   ├── common/         # Composants génériques
│   ├── forms/          # Composants de formulaires
│   └── layout/         # Composants de mise en page
├── pages/              # Pages de l'application
├── context/            # Context API pour l'état global
├── hooks/              # Hooks personnalisés
├── services/           # Services API
├── utils/              # Utilitaires
└── assets/             # Ressources statiques
\end{lstlisting}

\subsubsection{Gestion d'état}

Nous utilisons React Context API pour la gestion d'état global :
\begin{itemize}
    \item \textbf{AuthContext} : Gestion de l'authentification
    \item \textbf{AppContext} : État général de l'application
    \item \textbf{AdminContext} : État spécifique à l'administration
\end{itemize}

\subsection{Architecture backend (Node.js + Express)}

\subsubsection{Structure MVC}

\begin{lstlisting}[caption=Structure des dossiers backend]
backend/
├── controllers/        # Contrôleurs métier
├── models/            # Modèles de données
├── routes/            # Définition des routes
├── middleware/        # Middlewares d'authentification
├── services/          # Services métier
├── utils/             # Utilitaires
└── config/            # Configuration
\end{lstlisting}

\subsubsection{API RESTful}

Notre API suit les principes REST :

\begin{table}[h]
\centering
\begin{tabular}{|l|l|p{5cm}|}
\hline
\textbf{Endpoint} & \textbf{Méthode} & \textbf{Description} \\
\hline
/api/users & GET & Récupérer la liste des utilisateurs \\
\hline
/api/users & POST & Créer un nouvel utilisateur \\
\hline
/api/appointments & GET & Récupérer les rendez-vous \\
\hline
/api/appointments & POST & Créer un rendez-vous \\
\hline
/api/departments & GET & Récupérer les départements \\
\hline
\end{tabular}
\caption{Principaux endpoints de l'API}
\label{tab:api_endpoints}
\end{table}

\section{Sécurité et authentification}

\subsection{Stratégie de sécurité}

Notre approche sécuritaire couvre plusieurs aspects :

\subsubsection{Authentification}
\begin{itemize}
    \item Utilisation de JWT (JSON Web Tokens)
    \item Hachage des mots de passe avec bcrypt
    \item Expiration automatique des sessions
    \item Système de refresh tokens
\end{itemize}

\subsubsection{Autorisation}
\begin{itemize}
    \item Système de rôles (RBAC - Role-Based Access Control)
    \item Middleware de vérification des permissions
    \item Contrôle d'accès granulaire aux ressources
\end{itemize}

\subsubsection{Protection des données}
\begin{itemize}
    \item Chiffrement HTTPS en production
    \item Validation et sanitisation des entrées
    \item Protection contre les injections SQL/NoSQL
    \item Limitation du taux de requêtes (rate limiting)
\end{itemize}

\subsection{Conformité RGPD}

\begin{itemize}
    \item Consentement explicite pour la collecte de données
    \item Droit à l'effacement des données
    \item Portabilité des données
    \item Notification des violations de données
    \item Politique de confidentialité détaillée
\end{itemize}

\section{Diagrammes UML}

\subsection{Diagramme de cas d'usage}

\begin{figure}[h]
    \centering
    % \includegraphics[width=\textwidth]{images/use_case_diagram.png}
    \caption{Diagramme de cas d'usage du système}
    \label{fig:use_case}
\end{figure}

\subsection{Diagramme de séquence - Prise de rendez-vous}

\begin{figure}[h]
    \centering
    % \includegraphics[width=\textwidth]{images/sequence_appointment.png}
    \caption{Diagramme de séquence pour la prise de rendez-vous}
    \label{fig:sequence_appointment}
\end{figure}

\section{Conclusion}

Cette phase de conception nous a permis de définir une architecture robuste et scalable pour notre système de gestion des rendez-vous municipaux. Les choix technologiques effectués garantissent :

\begin{itemize}
    \item Une maintenabilité élevée du code
    \item Une expérience utilisateur optimale
    \item Une sécurité renforcée des données
    \item Une capacité d'évolution du système
\end{itemize}

Le chapitre suivant détaille la phase d'implémentation de cette conception.
