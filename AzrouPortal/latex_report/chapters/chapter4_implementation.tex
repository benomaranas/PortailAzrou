\chapter{Implémentation et développement}

\section{Introduction}

Ce chapitre détaille la phase d'implémentation de notre solution de gestion des rendez-vous municipaux. Nous présentons l'environnement de développement, les détails techniques de l'implémentation, ainsi que les défis rencontrés et les solutions apportées.

\section{Environnement de développement}

\subsection{Outils et technologies utilisés}

\begin{table}[h]
\centering
\begin{tabular}{|l|l|p{5cm}|}
\hline
\textbf{Catégorie} & \textbf{Outil} & \textbf{Version/Usage} \\
\hline
IDE & Visual Studio Code & Développement principal \\
\hline
Contrôle de version & Git + GitHub & Versionning et collaboration \\
\hline
Base de données & MongoDB Atlas & Base de données cloud \\
\hline
Test API & Postman & Test des endpoints \\
\hline
Design & Figma & Maquettage et prototypage \\
\hline
Déploiement & Vercel & Hébergement frontend \\
\hline
\end{tabular}
\caption{Environnement de développement}
\label{tab:dev_environment}
\end{table}

\subsection{Configuration du projet}

\subsubsection{Structure globale du projet}

\begin{lstlisting}[caption=Organisation des dossiers du projet]
prescripto-municipal/
├── frontend/           # Application React.js
├── backend/           # API Node.js + Express
├── admin/            # Interface d'administration
├── docs/             # Documentation
└── database/         # Scripts de base de données
\end{lstlisting}

\subsubsection{Configuration des dépendances}

\textbf{Frontend (package.json) :}
\begin{lstlisting}[language=JSON]
{
  "dependencies": {
    "react": "^18.2.0",
    "react-router-dom": "^6.8.1",
    "tailwindcss": "^3.2.7",
    "axios": "^1.3.4",
    "react-hot-toast": "^2.4.0"
  }
}
\end{lstlisting}

\textbf{Backend (package.json) :}
\begin{lstlisting}[language=JSON]
{
  "dependencies": {
    "express": "^4.18.2",
    "mongoose": "^7.0.3",
    "bcrypt": "^5.1.0",
    "jsonwebtoken": "^9.0.0",
    "nodemailer": "^6.9.1",
    "multer": "^1.4.5-lts.1"
  }
}
\end{lstlisting}

\section{Implémentation du backend}

\subsection{Configuration de la base de données}

\subsubsection{Connexion MongoDB}

\begin{lstlisting}[language=JavaScript, caption=Configuration MongoDB]
// config/mongodb.js
import mongoose from 'mongoose';

const connectDB = async () => {
  try {
    await mongoose.connect(process.env.MONGODB_URI, {
      useNewUrlParser: true,
      useUnifiedTopology: true,
    });
    console.log('MongoDB Connected');
  } catch (error) {
    console.error('Database connection error:', error);
    process.exit(1);
  }
};

export default connectDB;
\end{lstlisting}

\subsection{Modèles de données}

\subsubsection{Modèle User}

\begin{lstlisting}[language=JavaScript, caption=Modèle utilisateur]
// models/userModel.js
import mongoose from 'mongoose';

const userSchema = new mongoose.Schema({
  name: { 
    type: String, 
    required: true 
  },
  email: { 
    type: String, 
    required: true, 
    unique: true 
  },
  password: { 
    type: String, 
    required: true 
  },
  image: { 
    type: String, 
    default: '' 
  },
  address: {
    line1: { type: String, default: '' },
    line2: { type: String, default: '' }
  },
  gender: { 
    type: String, 
    default: 'Not Selected' 
  },
  dob: { 
    type: String, 
    default: '2000-01-01' 
  },
  phone: { 
    type: String, 
    default: '000000000' 
  }
}, { minimize: false });

const userModel = mongoose.model('user', userSchema);
export default userModel;
\end{lstlisting}

\subsubsection{Modèle Appointment}

\begin{lstlisting}[language=JavaScript, caption=Modèle rendez-vous]
// models/appointmentModel.js
import mongoose from 'mongoose';

const appointmentSchema = new mongoose.Schema({
  userId: { 
    type: String, 
    required: true 
  },
  docId: { 
    type: String, 
    required: true 
  },
  slotDate: { 
    type: String, 
    required: true 
  },
  slotTime: { 
    type: String, 
    required: true 
  },
  userData: { 
    type: Object, 
    required: true 
  },
  docData: { 
    type: Object, 
    required: true 
  },
  amount: { 
    type: Number, 
    required: true 
  },
  date: { 
    type: Number, 
    required: true 
  },
  cancelled: { 
    type: Boolean, 
    default: false 
  },
  payment: { 
    type: Boolean, 
    default: false 
  },
  isCompleted: { 
    type: Boolean, 
    default: false 
  }
});

const appointmentModel = mongoose.model('appointment', appointmentSchema);
export default appointmentModel;
\end{lstlisting}

\subsection{Contrôleurs}

\subsubsection{Contrôleur d'authentification}

\begin{lstlisting}[language=JavaScript, caption=Authentification utilisateur]
// controllers/userController.js
import bcrypt from 'bcrypt';
import jwt from 'jsonwebtoken';
import userModel from '../models/userModel.js';

// Connexion utilisateur
const loginUser = async (req, res) => {
  try {
    const { email, password } = req.body;
    
    const user = await userModel.findOne({ email });
    
    if (!user) {
      return res.json({
        success: false, 
        message: 'Utilisateur non trouvé'
      });
    }
    
    const isMatch = await bcrypt.compare(password, user.password);
    
    if (isMatch) {
      const token = jwt.sign(
        { id: user._id }, 
        process.env.JWT_SECRET
      );
      res.json({
        success: true, 
        token
      });
    } else {
      res.json({
        success: false, 
        message: 'Mot de passe incorrect'
      });
    }
  } catch (error) {
    console.log(error);
    res.json({
      success: false, 
      message: error.message
    });
  }
};

export { loginUser };
\end{lstlisting}

\subsubsection{Contrôleur de rendez-vous}

\begin{lstlisting}[language=JavaScript, caption=Gestion des rendez-vous]
// controllers/userController.js (suite)

// Réserver un rendez-vous
const bookAppointment = async (req, res) => {
  try {
    const { userId, docId, slotDate, slotTime } = req.body;
    
    const docData = await doctorModel.findById(docId).select('-password');
    
    if (!docData.available) {
      return res.json({
        success: false,
        message: 'Service non disponible'
      });
    }
    
    // Vérifier la disponibilité du créneau
    let slots_booked = docData.slots_booked;
    
    if (slots_booked[slotDate]) {
      if (slots_booked[slotDate].includes(slotTime)) {
        return res.json({
          success: false,
          message: 'Créneau non disponible'
        });
      } else {
        slots_booked[slotDate].push(slotTime);
      }
    } else {
      slots_booked[slotDate] = [];
      slots_booked[slotDate].push(slotTime);
    }
    
    const userData = await userModel.findById(userId).select('-password');
    
    delete docData.slots_booked;
    
    const appointmentData = {
      userId,
      docId,
      userData,
      docData,
      amount: docData.fees,
      slotTime,
      slotDate,
      date: Date.now()
    };
    
    const newAppointment = new appointmentModel(appointmentData);
    await newAppointment.save();
    
    // Mettre à jour les créneaux réservés
    await doctorModel.findByIdAndUpdate(docId, { slots_booked });
    
    res.json({
      success: true,
      message: 'Rendez-vous réservé avec succès'
    });
    
  } catch (error) {
    console.log(error);
    res.json({
      success: false,
      message: error.message
    });
  }
};

export { bookAppointment };
\end{lstlisting}

\subsection{Middleware d'authentification}

\begin{lstlisting}[language=JavaScript, caption=Middleware d'authentification]
// middleware/authUser.js
import jwt from 'jsonwebtoken';

const authUser = async (req, res, next) => {
  try {
    const { token } = req.headers;
    
    if (!token) {
      return res.json({
        success: false,
        message: 'Accès non autorisé'
      });
    }
    
    const token_decode = jwt.verify(token, process.env.JWT_SECRET);
    req.body.userId = token_decode.id;
    next();
    
  } catch (error) {
    console.log(error);
    res.json({
      success: false,
      message: error.message
    });
  }
};

export default authUser;
\end{lstlisting}

\section{Implémentation du frontend}

\subsection{Configuration React}

\subsubsection{Structure des composants}

\begin{lstlisting}[caption=Structure des composants React]
src/
├── components/
│   ├── Header.jsx          # En-tête avec navigation
│   ├── Footer.jsx          # Pied de page
│   ├── Navbar.jsx          # Barre de navigation
│   ├── TopDoctors.jsx      # Personnel municipal
│   ├── SpecialityMenu.jsx  # Menu des services
│   └── Banner.jsx          # Bannière d'accueil
├── pages/
│   ├── Home.jsx           # Page d'accueil
│   ├── Services.jsx       # Liste des services
│   ├── Doctors.jsx        # Personnel disponible
│   ├── Login.jsx          # Connexion
│   ├── About.jsx          # À propos
│   ├── Contact.jsx        # Contact
│   ├── MyProfile.jsx      # Profil utilisateur
│   └── MyAppointments.jsx # Mes rendez-vous
└── context/
    ├── AppContext.jsx     # Context principal
    └── ApiContext.jsx     # Context API
\end{lstlisting}

\subsection{Gestion d'état avec Context API}

\subsubsection{Context principal de l'application}

\begin{lstlisting}[language=JavaScript, caption=AppContext.jsx]
import { createContext, useEffect, useState } from "react";
import axios from 'axios';

export const AppContext = createContext();

const AppContextProvider = (props) => {
  const currencySymbol = 'MAD';
  const backendUrl = import.meta.env.VITE_BACKEND_URL;
  
  const [doctors, setDoctors] = useState([]);
  const [token, setToken] = useState(
    localStorage.getItem('token') 
      ? localStorage.getItem('token') 
      : false
  );
  const [userData, setUserData] = useState(false);
  
  // Récupérer les données des services/personnel
  const getDoctorsData = async () => {
    try {
      const { data } = await axios.get(
        backendUrl + '/api/doctor/list'
      );
      
      if (data.success) {
        setDoctors(data.doctors);
      } else {
        toast.error(data.message);
      }
    } catch (error) {
      console.log(error);
      toast.error(error.message);
    }
  };
  
  // Récupérer le profil utilisateur
  const loadUserProfileData = async () => {
    try {
      const { data } = await axios.get(
        backendUrl + '/api/user/get-profile',
        { headers: { token } }
      );
      
      if (data.success) {
        setUserData(data.userData);
      } else {
        toast.error(data.message);
      }
    } catch (error) {
      console.log(error);
      toast.error(error.message);
    }
  };
  
  const value = {
    doctors, getDoctorsData,
    currencySymbol, token, setToken,
    backendUrl, userData, setUserData,
    loadUserProfileData
  };
  
  useEffect(() => {
    getDoctorsData();
  }, []);
  
  useEffect(() => {
    if (token) {
      loadUserProfileData();
    } else {
      setUserData(false);
    }
  }, [token]);
  
  return (
    <AppContext.Provider value={value}>
      {props.children}
    </AppContext.Provider>
  );
};

export default AppContextProvider;
\end{lstlisting}

\subsection{Composants principaux}

\subsubsection{Page de prise de rendez-vous}

\begin{lstlisting}[language=JavaScript, caption=Composant de réservation]
// pages/Appointment.jsx
import React, { useContext, useEffect, useState } from 'react';
import { useNavigate, useParams } from 'react-router-dom';
import { AppContext } from '../context/AppContext';

const Appointment = () => {
  const { docId } = useParams();
  const { doctors, currencySymbol, backendUrl, token, getDoctorsData } = useContext(AppContext);
  
  const daysOfWeek = ['DIM', 'LUN', 'MAR', 'MER', 'JEU', 'VEN', 'SAM'];
  
  const [docInfo, setDocInfo] = useState(null);
  const [docSlots, setDocSlots] = useState([]);
  const [slotIndex, setSlotIndex] = useState(0);
  const [slotTime, setSlotTime] = useState('');
  
  const navigate = useNavigate();
  
  const fetchDocInfo = async () => {
    const docInfo = doctors.find(doc => doc._id === docId);
    setDocInfo(docInfo);
  };
  
  const getAvailableSlots = async () => {
    setDocSlots([]);
    
    // Logique de génération des créneaux
    let today = new Date();
    
    for (let i = 0; i < 7; i++) {
      let currentDate = new Date(today);
      currentDate.setDate(today.getDate() + i);
      
      let endTime = new Date(currentDate);
      endTime.setHours(21, 0, 0, 0);
      
      if (today.getDate() === currentDate.getDate()) {
        currentDate.setHours(
          currentDate.getHours() > 10 
            ? currentDate.getHours() + 1 
            : 10
        );
        currentDate.setMinutes(currentDate.getMinutes() > 30 ? 30 : 0);
      } else {
        currentDate.setHours(10);
        currentDate.setMinutes(0);
      }
      
      let timeSlots = [];
      
      while (currentDate < endTime) {
        let formattedTime = currentDate.toLocaleTimeString(
          [], { hour: '2-digit', minute: '2-digit' }
        );
        
        let day = currentDate.getDate();
        let month = currentDate.getMonth() + 1;
        let year = currentDate.getFullYear();
        
        const slotDate = day + "_" + month + "_" + year;
        const slotTime = formattedTime;
        
        const isSlotAvailable = 
          docInfo.slots_booked[slotDate] && 
          docInfo.slots_booked[slotDate].includes(slotTime) 
            ? false 
            : true;
        
        if (isSlotAvailable) {
          timeSlots.push({
            datetime: new Date(currentDate),
            time: formattedTime
          });
        }
        
        currentDate.setMinutes(currentDate.getMinutes() + 30);
      }
      
      setDocSlots(prev => ([...prev, timeSlots]));
    }
  };
  
  const bookAppointment = async () => {
    if (!token) {
      toast.warn('Connectez-vous pour réserver un rendez-vous');
      return navigate('/login');
    }
    
    try {
      const date = docSlots[slotIndex][0].datetime;
      
      let day = date.getDate();
      let month = date.getMonth() + 1;
      let year = date.getFullYear();
      
      const slotDate = day + "_" + month + "_" + year;
      
      const { data } = await axios.post(
        backendUrl + '/api/user/book-appointment', 
        { docId, slotDate, slotTime }, 
        { headers: { token } }
      );
      
      if (data.success) {
        toast.success(data.message);
        getDoctorsData();
        navigate('/my-appointments');
      } else {
        toast.error(data.message);
      }
    } catch (error) {
      console.log(error);
      toast.error(error.message);
    }
  };
  
  useEffect(() => {
    fetchDocInfo();
  }, [doctors, docId]);
  
  useEffect(() => {
    if (docInfo) {
      getAvailableSlots();
    }
  }, [docInfo]);
  
  return docInfo && (
    <div>
      {/* Interface de réservation */}
      <div className='flex flex-col sm:flex-row gap-4'>
        <div>
          <img className='bg-primary w-full sm:max-w-72 rounded-lg' src={docInfo.image} alt="" />
        </div>
        
        <div className='flex-1 border border-gray-400 rounded-lg p-8 py-7 bg-white mx-2 sm:mx-0 mt-[-80px] sm:mt-0'>
          <p className='flex items-center gap-2 text-2xl font-medium text-gray-900'>
            {docInfo.name}
          </p>
          <div className='flex items-center gap-2 text-sm mt-1 text-gray-600'>
            <p>{docInfo.degree} - {docInfo.speciality}</p>
            <button className='py-0.5 px-2 border text-xs rounded-full'>{docInfo.experience}</button>
          </div>
          
          <div>
            <p className='flex items-center gap-1 text-sm font-medium text-gray-900 mt-3'>
              À propos
            </p>
            <p className='text-sm text-gray-500 max-w-[700px] mt-1'>{docInfo.about}</p>
          </div>
          
          <p className='text-gray-500 font-medium mt-4'>
            Frais de service: <span className='text-gray-600'>{currencySymbol}{docInfo.fees}</span>
          </p>
        </div>
      </div>
      
      {/* Sélection des créneaux */}
      <div className='sm:ml-72 sm:pl-4 mt-4 font-medium text-gray-700'>
        <p>Créneaux disponibles:</p>
        <div className='flex gap-3 items-center w-full overflow-x-scroll mt-4'>
          {docSlots.length && docSlots.map((item, index) => (
            <div 
              onClick={() => setSlotIndex(index)} 
              className={`text-center py-6 min-w-16 rounded-full cursor-pointer ${slotIndex === index ? 'bg-primary text-white' : 'border border-gray-200'}`} 
              key={index}
            >
              <p>{item[0] && daysOfWeek[item[0].datetime.getDay()]}</p>
              <p>{item[0] && item[0].datetime.getDate()}</p>
            </div>
          ))}
        </div>
        
        <div className='flex items-center gap-3 w-full overflow-x-scroll mt-4'>
          {docSlots.length && docSlots[slotIndex].map((item, index) => (
            <p 
              onClick={() => setSlotTime(item.time)} 
              className={`text-sm font-light flex-shrink-0 px-5 py-2 rounded-full cursor-pointer ${item.time === slotTime ? 'bg-primary text-white' : 'text-gray-400 border border-gray-300'}`} 
              key={index}
            >
              {item.time.toLowerCase()}
            </p>
          ))}
        </div>
        
        <button 
          onClick={bookAppointment} 
          className='bg-primary text-white text-sm font-light px-14 py-3 rounded-full my-6'
        >
          Réserver un rendez-vous
        </button>
      </div>
    </div>
  );
};

export default Appointment;
\end{lstlisting}

\section{Interface d'administration}

\subsection{Tableau de bord administrateur}

\begin{lstlisting}[language=JavaScript, caption=Dashboard admin]
// admin/src/pages/Admin/Dashboard.jsx
import React, { useContext, useEffect } from 'react';
import { AdminContext } from '../../context/AdminContext';

const Dashboard = () => {
  const { 
    aToken, dashData, getDashData, 
    cancelAppointment 
  } = useContext(AdminContext);
  
  useEffect(() => {
    if (aToken) {
      getDashData();
    }
  }, [aToken]);
  
  return dashData && (
    <div className='m-5'>
      <div className='flex flex-wrap gap-3'>
        <div className='flex items-center gap-2 bg-white p-4 min-w-52 rounded border-2 border-gray-100 cursor-pointer hover:scale-105 transition-all'>
          <img className='w-14' src={assets.doctor_icon} alt="" />
          <div>
            <p className='text-xl font-semibold text-gray-600'>
              {dashData.doctors}
            </p>
            <p className='text-gray-400'>Personnel</p>
          </div>
        </div>
        
        <div className='flex items-center gap-2 bg-white p-4 min-w-52 rounded border-2 border-gray-100 cursor-pointer hover:scale-105 transition-all'>
          <img className='w-14' src={assets.appointment_icon} alt="" />
          <div>
            <p className='text-xl font-semibold text-gray-600'>
              {dashData.appointments}
            </p>
            <p className='text-gray-400'>Rendez-vous</p>
          </div>
        </div>
        
        <div className='flex items-center gap-2 bg-white p-4 min-w-52 rounded border-2 border-gray-100 cursor-pointer hover:scale-105 transition-all'>
          <img className='w-14' src={assets.patients_icon} alt="" />
          <div>
            <p className='text-xl font-semibold text-gray-600'>
              {dashData.patients}
            </p>
            <p className='text-gray-400'>Citoyens</p>
          </div>
        </div>
      </div>
      
      <div className='bg-white'>
        <div className='flex items-center gap-2.5 px-4 py-4 mt-10 rounded-t border'>
          <img className='w-5' src={assets.list_icon} alt="" />
          <p className='font-semibold'>Derniers rendez-vous</p>
        </div>
        
        <div className='pt-4 border border-t-0'>
          {dashData.latestAppointments.map((item, index) => (
            <div className='flex items-center px-6 py-3 gap-3 hover:bg-gray-50' key={index}>
              <img className='rounded-full w-10' src={item.docData.image} alt="" />
              <div className='flex-1 text-sm'>
                <p className='text-gray-800 font-medium'>{item.docData.name}</p>
                <p className='text-gray-600'>{item.slotDate}</p>
              </div>
              {item.cancelled 
                ? <p className='text-red-400 text-xs font-medium'>Annulé</p>
                : item.isCompleted 
                  ? <p className='text-green-500 text-xs font-medium'>Terminé</p>
                  : <img onClick={() => cancelAppointment(item._id)} className='w-10 cursor-pointer' src={assets.cancel_icon} alt="" />
              }
            </div>
          ))}
        </div>
      </div>
    </div>
  );
};

export default Dashboard;
\end{lstlisting}

\section{Défis techniques et solutions}

\subsection{Gestion des créneaux horaires}

\textbf{Défi :} Gérer les créneaux disponibles en temps réel pour éviter les doubles réservations.

\textbf{Solution :} Implémentation d'un système de verrouillage optimiste avec vérification côté serveur.

\subsection{Notifications email}

\textbf{Défi :} Envoi fiable des notifications de confirmation et de rappel.

\textbf{Solution :} Utilisation de Nodemailer avec un système de file d'attente pour les envois différés.

\begin{lstlisting}[language=JavaScript, caption=Service de notification]
// services/emailService.js
import nodemailer from 'nodemailer';

const transporter = nodemailer.createTransporter({
  service: 'gmail',
  auth: {
    user: process.env.EMAIL_USER,
    pass: process.env.EMAIL_PASS
  }
});

export const sendConfirmationEmail = async (userEmail, appointmentData) => {
  const mailOptions = {
    from: process.env.EMAIL_USER,
    to: userEmail,
    subject: 'Confirmation de rendez-vous - Commune d\'Azrou',
    html: `
      <h2>Confirmation de votre rendez-vous</h2>
      <p>Cher(e) ${appointmentData.userData.name},</p>
      <p>Votre rendez-vous est confirmé :</p>
      <ul>
        <li>Service: ${appointmentData.docData.name}</li>
        <li>Date: ${appointmentData.slotDate}</li>
        <li>Heure: ${appointmentData.slotTime}</li>
      </ul>
      <p>Merci de votre confiance.</p>
    `
  };
  
  return transporter.sendMail(mailOptions);
};
\end{lstlisting}

\subsection{Sécurisation des données}

\textbf{Défi :} Protection des données personnelles des citoyens.

\textbf{Solution :} Chiffrement des données sensibles et mise en place d'un système d'audit.

\section{Conclusion}

La phase d'implémentation nous a permis de concrétiser la conception en une application fonctionnelle. Les principales réalisations incluent :

\begin{itemize}
    \item Backend robuste avec API RESTful
    \item Interface utilisateur responsive et intuitive
    \item Système d'administration complet
    \item Gestion sécurisée des données
    \item Système de notifications automatiques
\end{itemize}

Le chapitre suivant présente la phase de tests et de validation de notre solution.
