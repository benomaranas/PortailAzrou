\chapter{Introduction générale}

\section{Contexte du projet}

Dans un monde de plus en plus connecté, la digitalisation des services publics devient une nécessité pour améliorer l'efficacité administrative et la satisfaction des citoyens. Les communes, en tant que collectivités territoriales les plus proches des citoyens, jouent un rôle crucial dans cette transformation numérique.

La commune d'Azrou, située dans la région de Fès-Meknès au Maroc, dessert une population diversifiée avec des besoins variés en termes de services municipaux. Actuellement, la prise de rendez-vous pour les services municipaux se fait principalement par déplacement physique ou par téléphone, ce qui peut entraîner des files d'attente, des pertes de temps et une inefficacité dans la gestion des ressources.

\section{Problématique}

La gestion traditionnelle des rendez-vous dans les services municipaux présente plusieurs défis :

\begin{itemize}
    \item \textbf{Accessibilité limitée} : Les citoyens doivent se déplacer physiquement ou appeler pendant les heures d'ouverture
    \item \textbf{Gestion manuelle} : Le personnel administratif doit gérer les rendez-vous manuellement, ce qui peut conduire à des erreurs et des doubles réservations
    \item \textbf{Manque de transparence} : Les citoyens n'ont pas de visibilité sur la disponibilité des créneaux
    \item \textbf{Inefficacité des ressources} : Difficultés dans la planification et l'optimisation des ressources humaines
    \item \textbf{Suivi difficile} : Absence d'outils de suivi et d'analyse des demandes de services
\end{itemize}

\section{Objectifs du projet}

\subsection{Objectif principal}

Développer une application web moderne et intuitive permettant la gestion digitalisée des rendez-vous pour les services municipaux de la commune d'Azrou.

\subsection{Objectifs spécifiques}

\begin{itemize}
    \item Permettre aux citoyens de prendre rendez-vous en ligne 24h/24 et 7j/7
    \item Fournir une interface d'administration complète pour le personnel municipal
    \item Implémenter un système de notifications automatiques
    \item Créer un tableau de bord analytique pour le suivi des performances
    \item Développer un module de signalement de problèmes communautaires
    \item Assurer la sécurité et la confidentialité des données des citoyens
\end{itemize}

\section{Méthodologie}

Ce projet suit une approche méthodologique structurée en plusieurs phases :

\begin{enumerate}
    \item \textbf{Analyse des besoins} : Étude des processus existants et identification des besoins
    \item \textbf{Conception} : Modélisation de la solution et définition de l'architecture
    \item \textbf{Développement} : Implémentation de l'application selon les spécifications
    \item \textbf{Tests} : Validation fonctionnelle et tests de performance
    \item \textbf{Déploiement} : Mise en production et formation des utilisateurs
\end{enumerate}

\section{Structure du mémoire}

Ce mémoire est organisé en cinq chapitres :

\begin{itemize}
    \item \textbf{Chapitre 1} : Présentation du contexte général et de l'organisme d'accueil
    \item \textbf{Chapitre 2} : Analyse des besoins et spécification des exigences
    \item \textbf{Chapitre 3} : Conception de la solution et modélisation
    \item \textbf{Chapitre 4} : Implémentation et développement
    \item \textbf{Chapitre 5} : Tests, validation et évaluation
\end{itemize}

Chaque chapitre contribue à démontrer la démarche scientifique adoptée pour répondre à la problématique identifiée et atteindre les objectifs fixés.
