\chapter{Contexte général et problématique}

\section{Présentation de l'organisme d'accueil}

\subsection{La Commune d'Azrou}

La Commune d'Azrou, située dans la région de Fès-Meknès au Maroc, constitue un centre urbain important du Moyen Atlas. Avec une population d'environ 60 000 habitants, elle représente un exemple type de collectivité territoriale confrontée aux défis de la modernisation administrative dans un contexte de digitalisation croissante des services publics.

\begin{table}[H]
\centering
\caption{Caractéristiques générales de la Commune d'Azrou}
\begin{tabular}{|l|l|}
\hline
\textbf{Caractéristique} & \textbf{Valeur} \\
\hline
Population & Environ 60 000 habitants \\
Superficie & 285 km² \\
Région & Fès-Meknès \\
Province & Ifrane \\
Statut & Commune urbaine \\
Services municipaux & 12 départements \\
Agents municipaux & Environ 150 personnes \\
\hline
\end{tabular}
\end{table}

\subsection{Structure administrative}

L'organisation administrative de la Commune d'Azrou s'articule autour de plusieurs départements spécialisés :

\begin{itemize}
\item \textbf{État Civil et Affaires Juridiques} : Gestion des actes d'état civil, légalisations, certificats administratifs
\item \textbf{Urbanisme et Aménagement} : Délivrance des autorisations de construire, suivi des projets urbains
\item \textbf{Affaires Financières} : Gestion budgétaire, recouvrement des taxes et impôts locaux
\item \textbf{Services Techniques} : Maintenance des infrastructures, gestion des espaces verts
\item \textbf{Services Sociaux} : Accompagnement social, programmes d'aide aux citoyens
\item \textbf{Hygiène et Environnement} : Gestion des déchets, contrôle sanitaire
\end{itemize}

\subsection{Mission et vision}

La Commune d'Azrou a pour mission principale de :
\begin{itemize}
\item Fournir des services publics de proximité de qualité
\item Assurer le développement économique et social du territoire
\item Maintenir et développer les infrastructures municipales
\item Promouvoir la participation citoyenne dans la vie locale
\end{itemize}

Sa vision s'oriente vers la création d'une commune moderne, connectée et à l'écoute de ses citoyens, s'appuyant sur les technologies numériques pour améliorer la qualité des services offerts.

\section{Contexte du projet}

\subsection{Évolution du paysage administratif marocain}

Le Maroc s'engage depuis plusieurs années dans une politique de modernisation de l'administration publique, notamment à travers la Stratégie Nationale de Développement du Gouvernement Électronique. Cette initiative vise à :

\begin{itemize}
\item Simplifier les procédures administratives
\item Améliorer la transparence et la redevabilité
\item Réduire les coûts de fonctionnement
\item Faciliter l'accès aux services publics
\end{itemize}

Dans ce contexte national, les collectivités territoriales sont encouragées à développer leurs propres solutions numériques pour répondre aux besoins spécifiques de leurs citoyens.

\subsection{Enjeux de la transformation numérique locale}

La transformation numérique au niveau communal présente plusieurs enjeux stratégiques :

\textbf{Enjeux citoyens :}
\begin{itemize}
\item Amélioration de l'accessibilité aux services municipaux
\item Réduction des délais d'attente et des déplacements
\item Transparence accrue dans le traitement des demandes
\item Disponibilité 24h/24 des services numériques
\end{itemize}

\textbf{Enjeux administratifs :}
\begin{itemize}
\item Optimisation des processus internes
\item Réduction de la charge administrative
\item Amélioration du suivi et du contrôle
\item Dématérialisation des documents
\end{itemize}

\textbf{Enjeux économiques :}
\begin{itemize}
\item Réduction des coûts opérationnels
\item Amélioration de l'efficacité énergétique
\item Développement de l'économie numérique locale
\item Attractivité territoriale renforcée
\end{itemize}

\section{Analyse de l'existant}

\subsection{Situation actuelle des services municipaux}

Avant le lancement de ce projet, la Commune d'Azrou fonctionnait selon un modèle traditionnel de prestation de services :

\begin{table}[H]
\centering
\caption{État des lieux des services municipaux}
\begin{tabular}{|p{4cm}|p{10cm}|}
\hline
\textbf{Aspect} & \textbf{Situation actuelle} \\
\hline
Accès aux services & Uniquement en présentiel aux guichets municipaux \\
Horaires & Limités aux heures d'ouverture administrative (8h-16h) \\
Procédures & Principalement papier avec archivage physique \\
Suivi des demandes & Pas de système de traçabilité centralisé \\
Paiements & Espèces ou chèques uniquement \\
Information citoyenne & Affichage physique et communication orale \\
\hline
\end{tabular}
\end{table}

\subsection{Problématiques identifiées}

L'analyse de l'existant a permis d'identifier plusieurs problématiques majeures :

\textbf{Du côté des citoyens :}
\begin{itemize}
\item Temps d'attente importants aux guichets municipaux
\item Nécessité de multiples déplacements pour certaines procédures
\item Manque d'information sur l'avancement des dossiers
\item Horaires d'ouverture contraignants pour les actifs
\item Difficultés d'accès pour les personnes à mobilité réduite
\end{itemize}

\textbf{Du côté de l'administration :}
\begin{itemize}
\item Gestion manuelle chronophage des dossiers
\item Risques de perte ou de détérioration des documents papier
\item Difficultés de recherche et d'archivage
\item Charge de travail importante pour les agents d'accueil
\item Manque d'outils statistiques pour le pilotage
\end{itemize}

\section{Problématique centrale}

\subsection{Formulation du problème}

Face aux constats établis, la problématique centrale du projet peut être formulée ainsi :

\begin{quote}
\textit{"Comment concevoir et développer une plateforme numérique intégrée qui permette à la Commune d'Azrou de moderniser ses services municipaux tout en améliorant significativement l'expérience utilisateur des citoyens et l'efficacité opérationnelle de l'administration ?"}
\end{quote}

\subsection{Questions de recherche}

Cette problématique principale se décline en plusieurs questions spécifiques :

\begin{enumerate}
\item Quelles sont les fonctionnalités prioritaires à implémenter pour répondre aux besoins les plus critiques ?
\item Comment garantir une interface utilisateur intuitive et accessible à tous les profils de citoyens ?
\item Quelle architecture technique adopter pour assurer la scalabilité et la maintenabilité de la solution ?
\item Comment intégrer les processus métier existants dans la nouvelle plateforme numérique ?
\item Quelles mesures de sécurité mettre en place pour protéger les données personnelles des citoyens ?
\end{enumerate}

\section{Objectifs du projet}

\subsection{Objectif général}

Développer une plateforme web complète et moderne permettant la dématérialisation des principaux services municipaux de la Commune d'Azrou, avec pour finalité l'amélioration de l'expérience citoyenne et l'optimisation des processus administratifs.

\subsection{Objectifs spécifiques}

\textbf{Objectifs fonctionnels :}
\begin{itemize}
\item Créer un système de prise de rendez-vous en ligne multi-services
\item Développer une interface de suivi des demandes en temps réel
\item Implémenter un module de paiement électronique sécurisé
\item Mettre en place un système de déclaration d'incidents
\item Créer un portail d'information municipale actualisé
\end{itemize}

\textbf{Objectifs techniques :}
\begin{itemize}
\item Développer une architecture scalable et maintenable
\item Assurer la compatibilité multi-navigateurs et multi-appareils
\item Garantir des performances optimales et une haute disponibilité
\item Implémenter des mécanismes de sécurité robustes
\item Prévoir l'évolutivité future de la plateforme
\end{itemize}

\textbf{Objectifs qualité :}
\begin{itemize}
\item Respecter les standards d'accessibilité web (WCAG)
\item Assurer une expérience utilisateur cohérente et intuitive
\item Garantir la fiabilité et la robustesse de la solution
\item Documenter exhaustivement le code et les processus
\item Prévoir des mécanismes de sauvegarde et de récupération
\end{itemize}

\section{Contraintes et limites}

\subsection{Contraintes techniques}

\begin{itemize}
\item Utilisation des technologies web modernes (React.js, Node.js)
\item Compatibilité avec l'infrastructure informatique existante
\item Respect des normes de sécurité des données personnelles
\item Optimisation pour les connexions internet variables
\end{itemize}

\subsection{Contraintes temporelles}

\begin{itemize}
\item Durée limitée du projet de fin d'études (6 mois)
\item Phases de développement et de test à réaliser dans les délais impartis
\item Nécessité de livrer une solution fonctionnelle et démonstrable
\end{itemize}

\subsection{Contraintes budgétaires}

\begin{itemize}
\item Utilisation privilégiée d'outils et frameworks open source
\item Optimisation des coûts de développement et de déploiement
\item Solutions d'hébergement adaptées aux moyens de la commune
\end{itemize}

\section{Méthodologie de travail}

\subsection{Approche méthodologique}

Le projet adopte une approche méthodologique mixte combinant :

\begin{itemize}
\item \textbf{Analyse préalable approfondie} : Étude des besoins et des contraintes
\item \textbf{Développement agile} : Itérations courtes avec validation régulière
\item \textbf{Prototypage rapide} : Tests précoces des interfaces utilisateur
\item \textbf{Validation continue} : Feedback des parties prenantes à chaque étape
\end{itemize}

\subsection{Outils et méthodes}

\begin{table}[H]
\centering
\caption{Outils et méthodes utilisés dans le projet}
\begin{tabular}{|l|l|}
\hline
\textbf{Phase} & \textbf{Outils/Méthodes} \\
\hline
Analyse & Interviews, questionnaires, observation \\
Conception & UML, maquettage, architecture logicielle \\
Développement & VS Code, Git, frameworks modernes \\
Tests & Tests unitaires, tests d'intégration \\
Documentation & LaTeX, diagrammes techniques \\
\hline
\end{tabular}
\end{table}

Cette approche méthodologique structurée permet d'assurer la qualité et la cohérence du développement tout en maintenant une flexibilité d'adaptation aux besoins évolutifs du projet.
