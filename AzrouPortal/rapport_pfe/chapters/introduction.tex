\chapter*{Introduction générale}
\addcontentsline{toc}{chapter}{Introduction générale}

À l'ère du numérique, la transformation digitale des administrations publiques est devenue une nécessité incontournable pour répondre aux attentes croissantes des citoyens en matière de qualité de service, d'accessibilité et de transparence. Les collectivités territoriales, en première ligne de la relation avec les administrés, se trouvent particulièrement concernées par cette évolution vers la dématérialisation des procédures administratives.

La Commune d'Azrou, consciente de ces enjeux contemporains, s'inscrit dans cette démarche de modernisation en initiant le développement d'une plateforme numérique complète destinée à faciliter et améliorer l'accès de ses citoyens aux services municipaux. Cette initiative s'inscrit dans une vision globale de transformation numérique visant à optimiser l'efficacité administrative tout en renforçant la satisfaction des usagers.

\section*{Contexte du projet}

Dans le cadre de mon projet de fin d'études à l'École Supérieure d'Ingénierie en Sciences Appliquées, j'ai eu l'opportunité de concevoir et développer une solution technologique répondant aux besoins spécifiques de la Commune d'Azrou. Ce projet s'articule autour de la création d'une plateforme web complète permettant aux citoyens d'accéder aux services municipaux de manière dématérialisée, tout en offrant aux agents municipaux les outils nécessaires à une gestion efficace des demandes.

\section*{Problématique}

La gestion traditionnelle des services municipaux présente plusieurs défis majeurs : files d'attente prolongées, horaires d'ouverture contraignants, procédures papier chronophages, manque de transparence dans le suivi des demandes, et difficultés d'accès pour certaines catégories de citoyens. Ces problématiques génèrent une insatisfaction croissante des usagers et une charge administrative importante pour le personnel municipal.

Comment développer une solution numérique intégrée qui permette de moderniser l'ensemble des services municipaux tout en garantissant une expérience utilisateur optimale et une gestion administrative efficace ?

\section*{Objectifs du projet}

Ce projet vise plusieurs objectifs stratégiques :

\textbf{Objectifs principaux :}
\begin{itemize}
\item Développer une plateforme web responsive accessible 24h/24 et 7j/7
\item Implémenter un système de prise de rendez-vous en ligne intuitif
\item Créer des interfaces d'administration pour la gestion des services municipaux
\item Assurer l'intégration des différents services dans une solution unifiée
\end{itemize}

\textbf{Objectifs secondaires :}
\begin{itemize}
\item Améliorer l'expérience utilisateur par une interface moderne et ergonomique
\item Réduire les délais de traitement des demandes administratives
\item Favoriser la transparence par le suivi en temps réel des demandes
\item Optimiser la charge de travail du personnel municipal
\end{itemize}

\section*{Méthodologie de travail}

L'approche méthodologique adoptée s'appuie sur les principes du développement agile avec une phase préliminaire d'analyse approfondie des besoins. La démarche comprend :

\begin{enumerate}
\item Une étude détaillée du contexte et des besoins des parties prenantes
\item L'analyse de l'existant et l'identification des exigences fonctionnelles
\item La conception architecturale de la solution
\item Le développement itératif avec tests réguliers
\item La validation et l'optimisation de la plateforme
\end{enumerate}

\section*{Structure du rapport}

Ce rapport s'organise autour de six chapitres principaux :

Le \textbf{premier chapitre} présente le contexte général du projet, l'organisme d'accueil et la problématique identifiée.

Le \textbf{deuxième chapitre} propose un état de l'art des solutions existantes et une analyse détaillée des besoins fonctionnels et non fonctionnels.

Le \textbf{troisième chapitre} expose la conception architecturale de la solution, les choix technologiques et la modélisation des données.

Le \textbf{quatrième chapitre} détaille l'implémentation de la plateforme, les fonctionnalités développées et les aspects techniques.

Le \textbf{cinquième chapitre} présente les tests réalisés, l'évaluation des performances et la validation de la solution.

Le \textbf{sixième chapitre} aborde le déploiement de la plateforme et propose des perspectives d'évolution.

Enfin, une conclusion générale synthétise les résultats obtenus et les apports du projet, tant sur le plan technique que sur l'impact attendu pour la Commune d'Azrou et ses citoyens.
